%\iffalse
\let\negmedspace\undefined
\let\negthickspace\undefined
\documentclass[journal,12pt,twocolumn]{IEEEtran}
\usepackage{cite}
\usepackage{circuitikz}
\usepackage{amsmath,amssymb,amsfonts,amsthm}
\usepackage{algorithmic}
\usepackage{graphicx}
\usepackage{textcomp}
\usepackage{xcolor}
\usepackage{txfonts}
\usepackage{listings}
\usepackage{enumitem}
\usepackage{mathtools}
\usepackage{gensymb}
\usepackage{comment}
\usepackage[breaklinks=true]{hyperref}
\usepackage{tkz-euclide} 
\usepackage{listings}
\usepackage{gvv}                                        
\def\inputGnumericTable{}                                 
\usepackage[latin1]{inputenc}                                
\usepackage{color}                                            
\newtheorem{theorem}{Theorem}[section]
\usepackage{array}                                            
\usepackage{longtable}                                       
\usepackage{calc}                                             
\usepackage{multirow}                                         
\usepackage{hhline}                                           
\usepackage{ifthen}                                           
\usepackage{lscape}
\usepackage{multicol}
\newtheorem{problem}{Problem}
\newtheorem{proposition}{Proposition}[section]
\newtheorem{lemma}{Lemma}[section]
\newtheorem{corollary}[theorem]{Corollary}
\newtheorem{example}{Example}[section]
\newtheorem{definition}[problem]{Definition}
\newcommand{\BEQA}{\begin{eqnarray}}
\newcommand{\EEQA}{\end{eqnarray}}
\newcommand{\define}{\stackrel{\triangle}{=}}
\theoremstyle{remark}
\newtheorem{rem}{Remark}
\begin{document}
\bibliographystyle{IEEEtran}
\vspace{3cm}
\title{Probability}
\author{ai24btech11024 PRAHLADHA% <-this % stops a space
}
\maketitle
\newpage
\bigskip
\renewcommand{\thefigure}{\theenumi}
\renewcommand{\thetable}{\theenumi}
\maketitle\Large{SectionC.MCQs with One Correct Answer}
\begin{enumerate}[start=16]\large
 
\item Two numbers are selected randomly from the set $S=\cbrak{1,2,3,4,5,6}$ without replacement one by one. The probability that minimum of the two numbers is less than 4 is   \hfill (2003S)
\begin{multicols}{4}
\begin{enumerate}
    \item $\frac{1}{15}$
    \item $\frac{14}{15}$
    \item $\frac{1}{5}$
    \item $\frac{4}{5}$
\end{enumerate}
\end{multicols}
\item If $P\brak{B}=\frac{3}{4},P\brak{A\cap B \cap \overline{C}}=\frac{1}{3}$ and $P\brak{\overline{A}\cap B\cap \overline{C}}=\frac{1}{3}$,then $P\brak{B\cap C}$ 

is \hfill (2003S)
\begin{multicols}{4}
\begin{enumerate}
    \item $\frac{1}{12}$
    \item $\frac{1}{6}$
    \item $\frac{1}{15}$
    \item $\frac{1}{9}$
\end{enumerate}
    
\end{multicols}
\item If three distinct numbers are chosen randomly from the first 100 natural numbers,then the probability that all three of them are divisible by both 2 and 3 is \hfill  (2004S)
\begin{multicols}{4}
\begin{enumerate}
\item $\frac{4}{25}$
\item $\frac{4}{35}$
\item $\frac{4}{33}$
\item $\frac{4}{1155}$
\end{enumerate}
    
\end{multicols}
\item A six fair dice is thrown until 1 comes, then the probability that 1 comes in even no. of trials is \hfill (2004S)
\begin{multicols}{4}
\begin{enumerate}
    \item $\frac{5}{11}$
    \item $\frac{5}{6}$
    \item $\frac{6}{11}$
    \item $\frac{1}{6}$
\end{enumerate}
\end{multicols}
\item One Indian and four American men and their wives are to be seated randomly around a circular table. Then the conditional probability that the Indian man is seated adjacent to his wife given that each other American man is seated adjacent to his wife is \hfill(2007-3 marks)
\begin{multicols}{4}
\begin{enumerate}
    \item $\frac{1}{2}$
    \item $\frac{1}{3}$
    \item $\frac{2}{5}$
    \item $\frac{1}{5}$
\end{enumerate}
\end{multicols}
\item let $E^{c}$ denote the complement of an event with $P\brak{G}>0$ and $P\brak{E\cap F\cap G}=0$. Then $P\brak{E^{c}\cap F^{c}|G}$ equals \hfill (2007-3 marks)
\begin{multicols}{2}
\begin{enumerate}
    \item $P\brak{E^{c}}+P\brak{F^{c}}$
    \item $P\brak{E^{c}}-P\brak{F^{c}}$
    \item $P\brak{E^{c}}-P\brak{F}$
    \item $P\brak{E}+P\brak{F^{c}}$    
\end{enumerate}
\end{multicols}
\item An experiment has 10 equally likely outcomes. Let A and B be non-empty events of the experiment. if A consists of 4 outcomes, the number of outcomes that B must have so that A and B are independent,is \hfill (2008)
\begin{multicols}{2}
\begin{enumerate}
    \item 2,4 or 8
    \item 3,6 or 9
    \item 4 or 8
    \item 5 or 10
\end{enumerate}
    
\end{multicols}
\item let $\omega$ be a complex cube root of unity with $\omega\neq 1$.A fair die is thrown three times.If $r_{1},r_{2}and r_{3}$ are the numbers obtained on the die, then the probability that $\omega^{r_{1}}+\omega^{r_{2}}+\omega^{r_{3}}=0$ is \hfill (2010)
\begin{multicols}{4}
\begin{enumerate}
    \item $\frac{1}{18}$
    \item $\frac{1}{9}$
    \item $\frac{2}{9}$
    \item $\frac{1}{36}$
\end{enumerate}
\end{multicols}
\item A signal which can be green or red with probability $\frac{4}{5}$ and $\frac{1}{5}$ respectively, is received by stationA and then transmitted to stationB. The probability of each station receiving the signal correctly is$\frac{3}{4}$. If the signal received at stationB is green, then the probability that the original signal was green is \hfill (2010)
\begin{multicols}{4}
\begin{enumerate}
    \item $\frac{3}{5}$
    \item $\frac{6}{7}$
    \item $\frac{20}{23}$
    \item $\frac{9}{20}$
\end{enumerate}
\end{multicols}
\item Four fair dice $D_{1},D_{2},D_{3} and D_{4}$; each having six faces numbered 1, 2, 3, 4, 5 and 6 are rolled simultaneously.
The probability that $D_{4}$ shows a number appearing on one of $D_{1},D_{2}andD_{3}$ is \hfill (2012)
\begin{multicols}{4}
\begin{enumerate}
    \item $\frac{91}{216}$
    \item $\frac{108}{216}$
    \item $\frac{125}{216}$
    \item $\frac{127}{216}$
\end{enumerate}
\end{multicols}
\item Three boys and girls stand in a queue. The probability that the number of boys ahead of every girl is at least one more than the number of girls ahead of her,is \hfill (JEE Adv.2014)
\begin{multicols}{4}
\begin{enumerate}
    \item $\frac{1}{2}$
    \item $\frac{1}{3}$
    \item $\frac{2}{3}$
    \item $\frac{3}{4}$
\end{enumerate}
\end{multicols}
\item A computer producing factory has only two plants $T_{1}$ and $T_{2}$.Plant $T_{1}$ produces 20\% and plant $T_{2}$ produces 80\% of the total computers produced.7\% of computers produced in the factory turn out to be defective.It is known that P(computers turn out to be defective given that is produced in plant $T_{1}$)=10P (computers turn out to be defective given that it is produced in plant $T_{2}$),

where P\brak{E} denotes the probability of an event E.A computer produced in the factory is randomly selected and it does not turn out to be defective. Then the probability that is produced in plant $T_{2}$ is \hfill (JEE Adv.2016)
\begin{multicols}{4}
\begin{enumerate}
    \item $\frac{36}{73}$
    \item $\frac{47}{79}$
    \item $\frac{78}{93}$
    \item $\frac{75}{83}$
\end{enumerate}
    
\end{multicols}






\item Three randomly chosen non-negative integers x,y and z are found to satisfy the equation $x+y+z=10$. Then the probability that z is even, is
\begin{multicols}{4}
\begin{enumerate}
    \item $\frac{36}{55}$
    \item $\frac{6}{11}$
    \item $\frac{1}{2}$
    \item $\frac{5}{11}$
\end{enumerate}
\end{multicols}
\end{enumerate}
\maketitle\Large{Section D.MCQs with One Correct Answer}
\begin{enumerate}
    \item If M and N are any two events, the probability that exactly one of them occurs is \hfill (1984-3Marks)
    \begin{enumerate}
    \item $P\brak{M}+P\brak{N}-2P\brak{M\cap N}$
    \item $P\brak{M}+P\brak{N}-P\brak{M\cap N}$
    \item $P\brak{M^{c}}+P\brak{N^{c}}-2P\brak{M^{c}\cap N^{c}}$
    \item $P\brak{M\cap N^{c}}+P\brak{M^{c}\cap N}$    
    \end{enumerate}
    \item A student appears for test I, II and III.The student is successful if he passes either in test I and II or tests I and III.The probabilities of the student passing in test I,II and III are p,q and $\frac{1}{2}$ respectively.If the probability that the student is successful is $\frac{1}{2}$,then \hfill (1986-2 Marks)
    \begin{multicols}{2}
    \begin{enumerate}
        \item $p=q=1$
        \item $p=q=\frac{1}{2}$
        \item $p=1,q=0$
        \item $p=1,q=\frac{1}{2}$
        \item none of these
    \end{enumerate}
        
    \end{multicols}
    
\end{enumerate}

\end{document}